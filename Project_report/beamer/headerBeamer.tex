\usetheme{mtec}
\setbeamercovered{transparent}

\usepackage[utf8]{inputenc}

%%%%%%%%%%
% FONTS %
%%%%%%%%%%

%% Default font: lmodern, doesn't require fontspec % solves some default warnings
%\usepackage[T1]{fontenc}
\usepackage{lmodern}            
%\usepackage{sfmath}        % Sans Serif Math, off by default

%% Protects fonts from Beamer screwing with them
%% http://tex.stackexchange.com/questions/10488/force-computer-modern-in-math-mode
\usefonttheme{professionalfonts}


%% XeLaTeX fonts: (comment out if you don't use XeLaTeX)

%% For advanced fonts: access local OS X fonts
%\usepackage[no-math]{fontspec}
%% This template uses typical OS X and Adobe fonts
%\defaultfontfeatures{Mapping=tex-text}  % This seems to be important for mapping glyphs properly

%\setmainfont{Gill Sans MT Pro Book}         % Beamer ignores "main font" in favor of sans font
%\setsansfont{Gill Sans MT Pro Book}         % This is the font that beamer will use by default
% \setmainfont{Gill Sans Light}     % Prettier, but harder to read

%\setbeamerfont{title}{family=\fontspec{Gill Sans MT Pro Book}}


%\newcommand{\handwriting}{\fontspec{augie}} % From Emerald City, free font
% \newcommand{\handwriting}{}   % If you prefer no special handwriting font or don't have augie

%% Gill Sans doesn't look very nice when boldfaced
%% This is a hack to use Helvetica instead
%% Usage: \textbf{\forbold some stuff}
%\newcommand{\forbold}{\fontspec{Helvetica}}
% \newcommand{\forbold}{} % if you want no special boldface


\usepackage{textcomp}
\usepackage{gensymb}
%%%%%%%%%%%%%%%%%%%%%%%%
% Usual LaTeX Packages %
%%%%%%%%%%%%%%%%%%%%%%%%
\usepackage{amsmath}
\usepackage{amssymb}
\usepackage{mathtools}
\usepackage{tabularx}
\usepackage{xcolor}
\usepackage{multirow}
\usepackage{bm}
\usepackage[separate-uncertainty=true]{siunitx}
\sisetup{detect-weight}

\usepackage[ngerman, english]{babel}
\usepackage{graphicx}
\usepackage{epstopdf}
\usepackage{epsfig}
\graphicspath{{../figures/}}
\usepackage{caption}
\usepackage{subcaption}
\captionsetup{compatibility=false}

\usepackage{tikz}
\usepackage{pgfplots}
\usetikzlibrary{shapes,arrows,3d,calc,fit,shadows}
\usepackage{tikz-3dplot}

%\usepackage{animate}
%\usepackage{media9}



\usepackage{hyperref}

\usepackage{xargs}

\makeatletter
\tikzoption{canvas is xy plane at z}[]{%
  \def\tikz@plane@origin{\pgfpointxyz{0}{0}{#1}}%
  \def\tikz@plane@x{\pgfpointxyz{1}{0}{#1}}%
  \def\tikz@plane@y{\pgfpointxyz{0}{1}{#1}}%
  \tikz@canvas@is@plane
}
\makeatother
\tikzset{xyp/.style={canvas is xy plane at z=#1}}
\tikzset{xzp/.style={canvas is xz plane at y=#1}}
\tikzset{yzp/.style={canvas is yz plane at x=#1}}

%%% Axes orientation for the Scanner Arrangement
\newcommandx*{\axisorientation}[7][1=220,2=-20,3=90,4=0.75,5=1,6=no]{
  \renewcommand{\xangle}{#1}
  \renewcommand{\yangle}{#2}
  \renewcommand{\zangle}{#3}

  \renewcommand{\xlength}{#4}
  \renewcommand{\ylength}{#5}
  \renewcommand{\zlength}{#7}
  \renewcommand{\lab}{#6}

  \pgfmathsetmacro{\xx}{\xlength*cos(\xangle)}
  \pgfmathsetmacro{\xy}{\xlength*sin(\xangle)}
  \pgfmathsetmacro{\yx}{\ylength*cos(\yangle)}
  \pgfmathsetmacro{\yy}{\ylength*sin(\yangle)}
  \pgfmathsetmacro{\zx}{\zlength*cos(\zangle)}
  \pgfmathsetmacro{\zy}{\zlength*sin(\zangle)}
}

\newcommandx*{\includetikz}[3][1=\linewidth,2=0.3\linewidth]{
 \newlength{\fwidth}
 \setlength{\fwidth}{#1}
 \newlength{\fheight}
 \setlength{\fheight}{#2}
 \input{#3}
 \global\let\fwidth\undefined
 \global\let\fheight\undefined
}
\newcommandx*{\includescaledtikz}[2][1=1]{
 \newcommand{\myscale}{#1}
 \input{#2}
}
%\usepackage[rgb]{xcolor}
\newcommand{\matr}[1]{\bm{#1}}     % ISO complying version
\newcommand{\vect}[1]{\bm{#1}}     % ISO complying version

\newcommand{\tfMat}[3]{{}^\mathrm{#1} \matr{#2}_\mathrm{#3}}
\newcommand{\invMat}[1]{\matr{#1}^{-1}}
