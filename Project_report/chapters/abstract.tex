\chapter{Abstract}
Electroencephalography (EEG) is used to record electrical activities of the brain using an electrode cap which is placed on the scalp. There are accurate methods developed to localize the electrodes, for example, using optical scanners, however, these methods are expensive and time-consuming. On the other hand, an inexpensive and faster solution is currently developed using a conventional camera and deep learning network. The development and evaluation of the method require a large amount of ground truth data for the electrode positions. In this work, a robotic system for the systematic acquisition of data for this purpose is developed. A phantom head wearing the EEG cap is mounted on to a robot and a head coordinate system using anatomical features of the phantom face is created. The position of all the electrodes in the EEG cap is recorded using a stylus with attached reflective markers and then transformed to the head coordinate system. The possible sources of error while using the camera, robot, and stylus for recording the positions of the electrodes have been examined. The data acquisition system can capture the electrode positions with respect to the head coordinate system with an accuracy of ±5mm. Three 30-45 seconds of video sequences with 30fps for three different EEG caps have been recorded with the system. This data can not only be used to train the convolutional neural networks (CNNs) and also to evaluate the performance of the camera-based electrode detection algorithm in general.